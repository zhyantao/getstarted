%%%%%%%%%%%%%%%%%%%%%%%%%%%%%%%%%%%%%%%%%
% Medium Length Professional CV
% LaTeX Template
% Version 2.0 (8/5/13)
%
% This template has been downloaded from:
% http://www.LaTeXTemplates.com
%
% Original author:
% Trey Hunner (http://www.treyhunner.com/)
%
% Important note:
% This template requires the cv.cls file to be in the same directory as the
% .tex file. The cv.cls file provides the resume style used for structuring the
% document.
%
%%%%%%%%%%%%%%%%%%%%%%%%%%%%%%%%%%%%%%%%%

%----------------------------------------------------------------------------------------
%	PACKAGES AND OTHER DOCUMENT CONFIGURATIONS
%----------------------------------------------------------------------------------------

\documentclass{cv} % Use the custom cv.cls style
\usepackage[dvipsnames]{xcolor}

\usepackage[left=0.75in,top=0.6in,right=0.75in,bottom=0.1in]{geometry} % Document margins
\newcommand{\tab}[1]{\hspace{.2667\textwidth}\rlap{#1}}
\newcommand{\itab}[1]{\hspace{0em}\rlap{#1}}
\name{Yantao Zhang} % Your name
\address{Xi'an, 710049 \\ Shaanxi, China} % Your address
%\address{123 Pleasant Lane \\ City, State 12345} % Your secondary addess (optional)
\address{Github: \href{https://github.com/zhyantao}{zhyantao} \\ Blog: \href{https://getstarted.readthedocs.io}{getstarted.rtfd.io}}
\address{(+86)156-3285-5317 \\ \href{mailto:zh6tao@gmail.com}{zh6tao@gmail.com}} % Your phone number and email


\renewenvironment{rSection}[1]{
\sectionskip
\textcolor{RoyalPurple}{\MakeUppercase{#1}}
\sectionlineskip
\hrule
\begin{list}{}{
\setlength{\leftmargin}{1.5em}
}
\item[]
}{
\end{list}
}



\begin{document}

%----------------------------------------------------------------------------------------
%	EDUCATION SECTION
%----------------------------------------------------------------------------------------

\begin{rSection}{Education}


  {\bf \href{https://www.xjtu.edu.cn}{Xi'an Jiaotong University}} \hfill {\em 2020 - 2023}
  \\ Software Engineering Masters in 2023 \hfill
  \\ School of Software Engineering, 2020 \hfill

  {\bf \href{https://www.ncst.edu.cn}{North China University of Science and Technology}} \hfill {\em 2016 - 2020}
  \\ BA in Electronic Information Engineering \hfill

  %Minor in Linguistics \smallskip \\
  %Member of Eta Kappa Nu \\
  %Member of Upsilon Pi Epsilon \\


\end{rSection}
%----------------------------------------------------------------------------------------
%	TECHNICAL STRENGTHS SECTION
%----------------------------------------------------------------------------------------

\begin{rSection}{Data Analytics Skills }

  \begin{tabular}{ @{} >{\bfseries}l @{\hspace{6ex}} l }
    Programming Languages & C/C++, Java, Python, SQL, Bash, MATLAB     \\
    Python Packages       & Pandas, Matplotlib, Numpy, Scipy, F2py, Psycopg2,  \\
                          & BeautifulSoup, Selenium, Spacepy, Davitpy, Jupyter \\
    Software \& Tools     & Linux, Git, HTML, LaTeX                  \\
  \end{tabular}

\end{rSection}

%----------------------------------------------------------------------------------------
%	WORK EXPERIENCE SECTION
%----------------------------------------------------------------------------------------

\begin{rSection}{Experience}

  \begin{rSubsection}{Insight Data Science}{June 2016 - August 2016}{Fellow}{}
    \item Developed a program to analyze Fitness Tracker data and compare results with weather information from Weather Underground, providing user with an analysis how to optimize their activity based on past habits.
    \item Wrote an API to web scrape Polar Loop Fitness Tracker data
    \item Data was obtained with Selenium, Data was managed through SQL, Pandas and Scipy were used for analysis, and Matplotlib was used for visualizations
  \end{rSubsection}


  %------------------------------------------------

  \begin{rSubsection}{NSF Graduate Research Fellow}{September 2013 - September 2016}{Graduate Reseach Assistant }{}
    \item Analyzed multiple time series satellite data sets ($>$ 1 Tb of data) to explore low energy ion loss in the inner Plasmasphere
    \item Developed an analytic model to demonstrate loss of ions from increased wave activity
    \item Developed algorithm to properly account for variability in the low energy ion pitch angle measurements when calculating partial ion densities.
  \end{rSubsection}

\end{rSection}


%	EXAMPLE SECTION
%----------------------------------------------------------------------------------------

\begin{rSection}{Achievements} \itemsep -2pt
  {Michigan Institute for Computational Discovery Fellow }\hfill {\em Spring 2015} \\
  {NSF GROW Fellowship Awardee}\hfill {\em Spring 2015} \\
  {Community Coordinated Modeling Center Research Winner} \hfill {\em Spring 2015} \\
  {NSF Graduate Research Fellowship Program Fellow}\hfill {\em Spring 2014}\\
  {Rackham Merit Fellow}\hfill {\em Fall 2013}\\
  {Template Developer for LaTeX} \hfill {\em September 2013 - Present} \\
  {Backpacker and Hiking Enthusiast - have climbed 7 $>$ 14,000 ft peaks}
\end{rSection}




\end{document}
